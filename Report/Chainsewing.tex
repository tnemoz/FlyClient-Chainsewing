    \section{\textit{Chain-sewing} attacks}
      A personal correspondence with Andrianna Polydouri and Dionysis Zindros showed that the Superblock protocol is prone to be attacked with what is called a \textit{chain-sewing} attack. Such an attack is applicable when the new protocol is deployed as a velvet fork.
      
      The idea is that the adversary can put arbitrary data in the interlink field. Hence, they can put interlink data that is invalid for honest miners but that may be considered valid by the verifier. Such an attack on \FC\ is described in \autoref{section:attack}.
    \section{Notations}
      \label{section:notations}
      We may begin with describing the notations that we will use throughout our analysis. We consider a blockchain \(\mathsf{C}\) consisting in \(n\) blocks. Amongst all nodes running this chain, a portion \(\alpha\) of them are malicious and work for a single adversary. To put things differently, we assume the existence of an adversary such that the ratio between their computational power and the total computational power of the network is \(\alpha\). We use a Python-like indexing of the blockchain \(\mathsf{C}\). That is, the genesis block is denoted \(\mathsf{C[}0\mathsf{]}\), the last block \(\mathsf{C[}-1\mathsf{]}\) or \(\mathsf{C[}n-1\mathsf{]}\) and we can represent the portion of the chain from block  \(i\) (inclusive) to block \(j\) (exclusive) with  \(\mathsf{C[}i\mathsf{:}j\mathsf{]}\). We use the same notations for denoting blocks in the adversary's fork, at the exception that we append \('\) to their name, like \(\mathsf{C[}i\mathsf{]}'\) for instance.    

      We work under the  \((c,L)\)-assumption, as described in \cite{\FCCite}. Fundamentally, it bounds the adversary computational power: if the adversary owns a fork of length \(L\) such that their own chain is as long as the main chain, it is impossible that a larger fraction than $c$ of these blocks valid with respect to the protocol rules. As a recall, \FC\ will always sample the last \(\ln\left(\frac{n}{L}\right)\) fraction of the blockchain's blocks. \(\delta\) is the number of blocks sampled with probability 1, that are the ones at the top of the chain.  

      Finally, two blocks are special regarding the \textit{chain-sewing} attack: what we call in the following the \emph{forking block}  \(f\) and the \emph{merging block} \(m\). Both these blocks are described in the next section. \autoref{table:notations} summarizes the notations used throughout the analysis.

      \begin{table}[ht]
        \centering
        \begin{tabular}{|c|c|}
          \hline
          Object & Representation\\
          \hline
          Blockchain & \(\mathsf{C}\)\\
          \hline
          Blockchain length &  \(n\)\\
          \hline
          Adversary's computational power &  \(\alpha\)\\
          \hline
          Block  \(i\) of the chain &  \(\mathsf{C[}i\mathsf{]}\)\\
          \hline
          Block \(i\) of the chain in the adversary's fork & \(\mathsf{C[}i\mathsf{]}'\)\\
          \hline
          Blocks \(i\) (inclusive) to  \(j\) (exclusive) of the chain &  \(\mathsf{C[}i\mathsf{:}j\mathsf{]}\)\\
          \hline
          \((c,L)\)-adversary assumption &  \((c,L)\)\\
          \hline
          Position of the forking block & \(f\)\\
          \hline
          Position of the merging block &  \(m\)\\
          \hline
          Number of blocks sampled with probability 1 at the end of the chain & \(\delta\)\\
          \hline
        \end{tabular}
        \caption{Notations used throughout the analysis}
        \label{table:notations}
      \end{table} 
      
    \section{Principle of a \textit{chain-sewing} attacks on \FC}
      \label{section:attack}
      Let us place ourselves within the Bitcoin backbone protocol with variable difficulty. Let us assume that the \FC\ protocol was implemented on a velvet fork in this context. In particular, this means that:

      \begin{itemize}
        \item every block header contains a reference to the previous block in the chain, since it has to be valid according to the old protocol, be it a Proof of Work or a Proof of Stake;
        \item it is possible for a prover to indicate a block as a legacy block. Otherwise, every miner who puts random data in the MMR root field will break the protocol as no proofs sampling this block would be accepted;
        \item a block header can contain arbitrary data in the MMR root field;
        \item if a legacy block is sampled, the prover must provide all its ancestors until the most recent upgraded block.
      \end{itemize}
      
      Let us assume that at a block \(f\) of the main chain \(\mathsf{C}\), which we will call the \textit{forking block} from now on, the adversary creates a fork. In the mean time, both the honest miners and potentially the adversary continue to mine on top of the main chain. In order to do so, the adversary splits its computational power so that they mine on top of the fork with a portion \(\beta\) of its computational power and uses the remaining to mine on top of the main chain. The adversary will behave according to the \FC\ protocol on both the fork and the main chain. Honest miners will continue to behave accordingly to the \FC\ protocol. The situation is represented on \autoref{figure:chainsewingattempt}. In this figure:
      
      \begin{itemize}
        \item black blocks are mined by the adversary and contain a MMR root;
        \item dashed arrows correspond to MMR link if it is different from the previous block header hash reference;
        \item snake arrows represents a portion of blocks, be they mined by the adversary or by a honest miner;
        \item every block is valid according to the old protocol rules.
      \end{itemize}
    
      \begin{figure}[ht]
        \centering
          \begin{tikzpicture}[decoration = {snake}]
            \node[block] (G) {};
            \node[below of=G] {\(\mathsf{C[}0\mathsf{]}\) };
            \node[block, right=of G] (c) {};
            \node[below of=c] {\(\mathsf{C[}f\mathsf{]}\) };
            \node[block, above right=of c,fill=black] (cprime1) {};
            \node[below of=cprime1] {\(\mathsf{C[}f+1{]}'\) };
            \node[block, below right=of c] (c1) {};
            \node[block, right=of cprime1,fill=black] (cprimei) {};
            \node[below of=cprimei] {\(\mathsf{C[}f+i\mathsf{]}'\) };
            \node[below of=c1] {\(\mathsf{C[}f+1\mathsf{]}\) };
            \node[block, right=of c1] (cj) {};
            \node[below of=cj] {\(\mathsf{C[}f+j\mathsf{]}'\) };
            \node[block, above right=of cj, fill=black] (cj1) {};
            \node[below of=cj1] {\(\mathsf{C[}m\mathsf{]}\) };
            \node[block, right=of cj1] (k) {};
            \node[below of=k] {\(\mathsf{C[}n-k\mathsf{]}\) };
            \node[block, right=of k] (N) {};
            \node[below of=N] {\(\mathsf{C}[-1\mathsf{]}\) };
            
            \path[draw, decorate] (G) -- (c);
            \path[draw] (cprime1.west) -- (c.north);
            \path[draw] (c1.west) -- (c.south);
            \path[draw, decorate] (cprime1) -- (cprimei);
            \path[draw, decorate] (c1) -- (cj);
            \path[draw, dashed, ->] (cj1.north) -- (cprimei.east);
            \path[draw] (cj1.south) -- (cj.east);
            \path[draw, decorate] (cj1) -- (k);
            \path[draw,decorate] (k) -- (N);
          \end{tikzpicture}
          \caption{A chainsewing attack attempt on \FC}
          \label{figure:chainsewingattempt}
        \end{figure}
      
Starting from \(\mathsf{m}=\mathsf{f+j+1}\), which we will call the \textit{merging block} from now on, every block mined by the adversary will contain a MMR root corresponding to the MMR where the portion \(\mathsf{C[}f+1:f+j+1\mathsf{]}\) has been replaced with the adversary's fork, and where all honest miner's blocks are considered as legacy blocks. Once in this situation, a number \(n-m>k\) of blocks are mined on top of \(\mathsf{C[}f+j+1\mathsf{]}\). In order to have, let us say, \(\mathsf{C[}f+1\mathsf{]}'\) accepted, the adversary must:
      
      \begin{itemize}
        \item convince the verifier that they hold a chain \(\mathsf{C}'\) that is as long as the main chain;
        \item provide the verifier with a MMR proof that \(\mathsf{C[}f+1\mathsf{]}'\) lies within \(\mathsf{C}'\).
      \end{itemize}
      
      Let us focus on the latter for now. As a recall, the adversary blocks and an honest miner's now have different MMR roots, and both considers the other's blocks as legacy blocks. For this reason, it is necessary for a prover to have the capacity to designate a sampled block as legacy. Otherwise, whenever an adversary's block is mined, the proof provided by an honest miner will fail.
      
      Hence, it is possible for the adversary to designate any honest miner's block as a legacy block, so that only adversarial blocks are sampled. Because of the way the adversary has built its MMR, they will succeed in proving the inclusion of \(\Cindex{f+1}'\) in the chain the adversary claims to have.
      
      However, the adversary still has to prove that the MMR root in \(\Cindex{f+1}'\) belongs to a chain of the same length as the main chain, that is to prove that the chain they claims to have is the longest chain they know. Indeed, two cases are possible:
      
      \begin{enumerate}
      \item The verifier only connects to the adversary.
      \item The verifier connects to at least one honest prover.
      \end{enumerate}
      
      The first case is actually trivial: since the verifier has no other information than the ones provided by the adversary, the adversary will succeed in proving that she holds the longest chain, since there is no other chain anyway.
      
      The second case is more difficult. An honest prover will tell the verifier that they hold a chain of length \(n\). Since longer chains will be verified first, the adversary also has to claim having a chain of length \(n\), while only having a chain of length \(n-j+i\). \FC\ has been built for preventing this very situation. Hence, the only way for the adversary to succeed is to set \(j\geqslant i\). The adversary wants however to include as less fake blocks as possible. Hence, they have to set \(j=i\) since they can't mine faster than the main chain. Indeed, achieving \(j=i\) while keeping minimal the number of fake blocks is already hard.  The best strategy is then to try to mine \(\Cindex{x}'\) while \(\Cindex{x}\) is mined on the main chain for some \(x\). If \(\Cindex{x}'\) is mined before \(\Cindex{x}\), the adversary begins to mine \(\Cindex{x+1}'\). If it is not, the adversary places a fake block at place \(x\) and begins to mine \(\Cindex{x+1}'\). Note that even if the adversary have some advance, they have to wait until the corresponding block is mined for mining the merging block. For simplicity, let us take \(j=i=1\). This corresponds to the situation in \autoref{figure:chainsewingattack}.
      
      \begin{figure}[ht]
        \centering
          \begin{tikzpicture}[decoration = {snake}]
            \node[block] (G) {};
            \node[below of=G] {\(\Cindex{0}\) };
            \node[block, right=of G] (c) {};
            \node[below of=c] {\(\Cindex{f}\) };
            \node[block, above right=of c, fill=black] (cprime1) {};
            \node[below of=cprime1] {\(\Cindex{f+1}'\) };
            \node[block, below right=of c] (c1) {};
            \node[below of=c1] {\( \Cindex{f+1}\) };
            \node[block, above right=of c1, fill=black] (c2) {};
            \node[below of=c2] {\(\Cindex{m}\) };
            \node[block, right=of c2] (k) {};
            \node[below of=k] {\(\Cindex{n-k}\) };
            \node[block, right=of k] (N) {};
            \node[below of=N] {\(\Cindex{-1}\) };
            
            \path[draw, decorate] (G) -- (c);
            \path[draw] (cprime1.west) -- (c.north);
            \path[draw] (c1.west) -- (c.south);
            \path[draw, dashed, ->] (c2.north) -- (cprime1.east);
            \path[draw] (c2.south) -- (c1.east);
            \path[draw, decorate] (c2) -- (k);
            \path[draw,decorate] (k) -- (N);
          \end{tikzpicture}
          \caption{A chainsewing attack attempt on \FC\ with a fork of length 1}
          \label{figure:chainsewingattack}
        \end{figure}
        
      Now, it is easy for the adversary to claim having a chain of length \(n\). Actually, they can even claim having a longer chain if they manage to mine a block at the top of the chain and by keeping it secret from the honest miners for a time. The advantage of doing this is that \FC\ will begin by the longer proof, that is the adversary's.
      
      However, is is crucial that \(\Cindex{m}\) is not sampled when proving this. Indeed, an inconsistency between its PoW (or more generally, its reference to the previous block) and its MMR would be revealed. Indeed, the verifier is able to know that \(\Cindex{f+1}'\) and \(\Cindex{m}\) are supposed to be adjacent, according to the MMR structure they deduced from \(n\), that the adversary had to provide. Since the adversary wants \(\Cindex{f+1}'\) to be verified, they will have to send it to the prover. Hence, if \(\Cindex{m}\) is sampled by the client, then an inconsistency between the MMR root and the previous block can be detected by the client.
      
      Note that this case is not actually described in the \FC\ paper. Hence, if \FC\ is deployed without taking this problem into account, the probability of succees, as computed in \autoref{chapter:probability} increases. 
      
      A solution for the adversary to avoid this is simply to wait for blocks being mined on top of the main chain. Indeed, the original design of \FC\ makes old blocks less-likely to be sampled. Hence, by doing so, it is high-likely that \(\Cindex{m}\) won't be sampled, and that no inconsistency will be detected when the adversary will send \(\Cindex{f+1}'\).
      
      In order to circumvent this problem, one may also try to introduce intermediary blocks, valid or not, between \(\Cindex{f+1}'\) and \(\Cindex{m}\). Indeed, the problem here is that since we want \(\Cindex{f+1}'\) accepted, we have to provide the client with it, and this block is somehow too close to \(\Cindex{m}\). Let us consider the situation depicted in \autoref{figure:chainsewingdouble}.
      
      \begin{figure}[ht]
        \centering
        \begin{tikzpicture}[decoration = {snake}]
          \node[block] (G) {};
          \node[below of=G] {\(\Cindex{0}\) };
          \node[block, right=of G] (c) {};
          \node[below of=c] {\(\Cindex{f}\) };
          \node[block, above right=of c,fill=black] (cprime1) {};
          \node[below of=cprime1] {\(\Cindex{f+1}'\) };
          \node[block, below right=of c] (c1) {};
          \node[block, right=of cprime1, fill=black] (cprime2) {};
          \node[below of=cprime2] {\(\Cindex{f+2}'\) };
          \node[below of=c1] {\(\Cindex{f+1}\) };
          \node[block, right=of c1] (c2) {};
          \node[below of=c2] {\(\Cindex{f+2}\) };
          \node[block, above right=of c2, fill=black] (c3) {};
          \node[below of=c3] {\(\Cindex{m}\) };
          \node[block, right=of c3] (k) {};
          \node[below of=k] {\(\Cindex{n-k}\) };
          \node[block, right=of k] (N) {};
          \node[below of=N] {\(\Cindex{-1}\) };
          
          \path[draw, decorate] (G) -- (c);
          \path[draw] (cprime1.west) -- (c.north);
          \path[draw] (c1.west) -- (c.south);
          \path[draw] (cprime1) -- (cprime2);
          \path[draw] (c1) -- (c2);
          \path[draw, dashed, ->] (c3.north) -- (cprime2.east);
          \path[draw] (c3.south) -- (c2.east);
          \path[draw, decorate] (c3) -- (k);
          \path[draw,decorate] (k) -- (N);
        \end{tikzpicture}
        \caption{A try to prevent the merging block sampling problem}
        \label{figure:chainsewingdouble}
      \end{figure}
        
      The adversary significantly increases its probablity of success by doing so, but this is also more difficult to setup. Indeed, several cases of sampling are possible. If neither \(\Cindex{m}\) nor \(\Cindex{f+2}'\) are sampled, there is no problem and the attack succeeds. If \(\Cindex{f+2}'\) is sampled without \(\Cindex{m}\), the attack also succeeds, since \(\Cindex{f+2}'\) is valid both from the previous block reference point of view and from the MMR root point of view. If both \(\Cindex{m}\) and \(\Cindex{f+2}'\) are sampled, the attack fails, since the inconsistency between the previous block reference of \(\Cindex{m}\) and the hash of \(\Cindex{f+2}'\) would then be revealed. Finally, if \(\Cindex{m}\) is the only one to be sampled, then it depends on whether \(m\) is odd. If \(m\) is even, then \(\Cindex{m}\) and \(\Cindex{f+2}'\) will share the same parent node in the MMR the adversary would have built. Hence, the hash of \(\Cindex{f+2}'\) would have to be provided in the MMR proof of the inclusion of \(\Cindex{m}\) within the chain. If \(m\) is odd however, they don't share the same parent node. Hence, the hash of \(\Cindex{f+2}'\) won't appear in the proof and the inconsistency wouldn't be detected.  

      Getting back to the \textit{chain-sewing attack}, the reason why this works on a velvet fork only is that an adversary is allowed to put some fake MMR root in a block header of the main chain. In particular, the following was outlined in the \FC\ paper: \enquote{once a malicious prover forks off from the honest chain, it cannot include any of the later honest blocks in its chain because the MMR root in those blocks would not match the chain} \cite{\FCCite}.
      
      In particular, the adversary is not forced to create a fork as long as the main chain, eventually creating fake blocks. We may note that this attack works as long as the fork created by the adversary is as long as the corresponding chain portion. The adversary can also include fake blocks in its fork to have a longer fork while sticking to this constraint. Even though this increases the probability of getting caught as every fake block sampled results in a failed proof, waiting long enough once the fork has been merged is enough for hoping that these blocks won't be sampled. Hence, the attack also works with longer forks.

